\documentclass[a4paper]{article}
\usepackage[T1]{fontenc}
\usepackage[utf8x]{inputenc}
\usepackage[margin=1.1in]{geometry}
\usepackage[english]{babel}
\usepackage{graphicx}
\usepackage{parskip}
\usepackage{float}

\usepackage{setspace} % increase interline spacing slightly
\setstretch{1.1}

\def \hfillx {\hspace*{-\textwidth} \hfill}

\title{
	Big Data Analytics \\
	Lab 02 - Wikipedia}
\author{
	Damien Rochat, Dorian Magnin, and Nelson Rocha \\
	Master of Science in Engineering \\
	University of Applied Sciences and Arts Western Switzerland}
\date{\today}

\begin{document}
	\maketitle
	
	\section{Results}
	Here is the number of articles where each programming language has been found. Note, the comparison is made by lowercase.

	\begin{table}[H]
		\begin{minipage}{0.5\textwidth}
			\centering

			\begin{tabular}{|l|l|l|}
				\hline
				\textbf{Rank}  & \textbf{Language}  & \textbf{Number of articles} \\ \hline
				1              & JavaScript         & 1'721                       \\ \hline
				2              & C\#                & 707                         \\ \hline
				3              & Java               & 618                         \\ \hline
				4              & CSS                & 400                         \\ \hline
				5              & C++                & 335                         \\ \hline
				6              & Python             & 315                         \\ \hline
				7              & MATLAB             & 307                         \\ \hline
				8              & PHP                & 302                         \\ \hline
				9              & Perl               & 167                         \\ \hline
				10             & Ruby               & 125                         \\ \hline
				11             & Haskell            & 56                          \\ \hline
				12             & Objective-C        & 47                          \\ \hline
				13             & Scala              & 44                          \\ \hline
				14             & Clojure            & 26                          \\ \hline
				15             & Groovy             & 26                          \\ \hline
			\end{tabular}
			\caption{Wikipedia lab ranking}

		\end{minipage}
		\hfillx
		\begin{minipage}{0.5\textwidth}
			\centering

			\begin{tabular}{|l|l|l|}
				\hline
				\textbf{Rank}  & \textbf{Language} \\ \hline
				1              & JavaScript        \\ \hline
				2              & Java              \\ \hline
				3              & Python            \\ \hline
				4              & PHP               \\ \hline
				5              & C\#               \\ \hline
				6              & C++               \\ \hline
				7              & CSS               \\ \hline
				8              & Ruby              \\ \hline
				9              & C                 \\ \hline
				10             & Objective-C       \\ \hline
				11             & Swift             \\ \hline
				12             & Scala             \\ \hline
				13             & Shell             \\ \hline
				14             & Go                \\ \hline
				15             & R                 \\ \hline
				16             & TypeScript        \\ \hline
				17             & PowerShell        \\ \hline
				18             & Perl              \\ \hline
				19             & Haskell           \\ \hline
				20             & Lua               \\ \hline
			\end{tabular}
			\caption{RedMonk top-20 ranking (June 2018)}

		\end{minipage}
	\end{table}

	Except for JavaScript, large winner in both cases, the Wikipedia ranking doesn't match RedMonk one.
	However, the top-5 of the two rankings is almost the same, in different order.

	\pagebreak

	\section{Performances}
	
	\begin{table}[H]
		\centering

		\begin{tabular}{|l|l|l|}
			\hline
			\textbf{Iteration}           & \textbf{Time} \\ \hline
			Naive ranking                & 32'798 ms     \\ \hline
			Ranking using inverted index & 13'250 ms     \\ \hline
			Ranking using reduceByKey    & 10'607 ms     \\ \hline
		\end{tabular}
		\caption{Processing time comparison}
	\end{table}
	
	TODO: explain...

\end{document}
